\documentstyle[12pt]{article}
\def\P{\mbox{P}}
\def\G{\Gamma}
\def\t{\theta}
\def\a{\alpha}
\def\E{\mbox{E}}
\parindent=0in
%\parskip=5mm
\pagestyle{empty}

\begin{document}

\begin{center}
{\bf
Bayesian Statistics Examples

\smallskip

Maximizing Expected Utility/ Minimizing Expected Loss.
}
\smallskip

\end{center}

\bigskip
\begin{enumerate}
\item {\it You are a competitor on a quiz show. So far, you have won five thousand dollars, and are given the choice between i) leaving with the five thousand dollars; ii) attempting a bonus question, which, if you answer correctly, will give you a Toyota Prius car. Which do you choose?}

{\bf Answer 1:} You have seen the quiz show before, and the bonus questions are quite hard: you think
you are able to answer them correctly about half the time. That is, your personal probability that
you will answer the bonus question correctly is 0.5. On the other hand, you have always wanted
a Prius, and your current car is on its last legs so you are going to buy a new one soon
anyway. The Prius costs more than 20,000 dollars. So, treating dollars as utility, the expected utility
for i) is 5,000 dollars, and for ii) is 10,000 dollars, so you choose ii).

{\bf Answer 2:} You are happy with your car, and don't really like the Prius. Of course, you could always
sell the Prius. Perhaps you would get 18,500 for it. But it would be a bit of work, so maybe
the Prius is worth 18,000 to you when you've accounted for that. Also,
although you think you could answer previous bonus questions right about half the time, you
are feeling unlucky today, so your probability that you will get today's question right is more like 0.25.
So your utility for i) is still 5,000 dollars, but for ii) it is 0.25*18,000 = 4,500. You choose i).

\medskip

\item {\it (Purchasing Insurance) You are buying a new Ipod at Best Buy. It costs 300 dollars,
and comes with a 1 year warranty.
They ask you if you would like to purchase the extended 3-year warranty for 30 dollars. Do you?}

{\bf Possible Answer: } You know that they must be expecting to make money on the warranty, so you figure
that it is used by fewer than one person in ten. If you buy the warranty, that will cost you 30 dollars.
If you do not buy the warranty, then if your ipod breaks during years 2 or 3 you will have to replace
it. But costs are dropping, and things always get better, so you will presumably spend less than 300
to replace it for something better. So you figure your utility for buying it is -30, and your
utility for not buying it is more than (-300)*0.1 = -30. So you don't buy it.

\medskip

\item {\it (Purchasing Insurance 2). You have just bought a new house, for 500,000 dollars.
To insure the house against fire for one year costs 1,000 dollars. Do you do it?}

{\bf Possible Answer:}  Fire is pretty unlikely. Indeed, you know that they expect to make
money on insurance, so presumably it affects less than 1 in 500 houses per year. On the other
hand, if your house burned down you would lose everything, and, to you, 
losing your house is unthinkable: much more than 500 times worse than losing 1,000 dollars.
So you buy the insurance. (Note: Of course, in most cases the people lending you money to buy a house
insist you insure it!)


\item {\it (Point Estimation). In current genetic studies, the most commonly-used type of marker is called a SNP (Single Nucleotide Polymorphism). Each SNP is a single position in the genome, where different copies of the genome carry one of two different bases. For example, at a C/G SNP, some genome copies carry the C, and others have a G. Since we each have two copies of our genome (one from mother,
or one from father), at a C/G SNP each of us will have either 0, 1 or 2 Cs. That is our "genotype" can be thought of as a 0, 1 or 2 variable.

Suppose now that we have a method for measuring genotypes. This method acknowledges
that the measurement may have errors, so instead of just giving a genotype for each individual,
it instead gives probabilities $p_0,p_1$ and $p_2$ for the genotype to be 0,1 or 2.
Given these probabilities, what would you report to be the genotype?}

{\bf Answer 1 (0-1 loss; mode):} You decide that if the genotype is incorrect, you will lose 1 unit, whereas
if it is correct, you lose 0. This is referred to as 0-1 loss, and can be written as
$$L(\hat{g}; g) = I(\hat{g} \neq g),$$
where $L(\hat{g}; g)$ denotes the loss (``consequence") you suffer if you report $\hat{g}$ (``action")
when the truth is $g$ (``Event"). 
Note that 0-1 loss assumes that, when you are wrong, it does not matter
{\it how} wrong you are. 

If you report anything but 0, 1 or 2 as your genotype you are certain
to be wrong, so your expected loss would be 1. If you report $g \in \{0,1,2\}$ then your expected
loss is $1-p_g$. To minimize your expected loss you report $\hat{g} = \arg \max p_g$. That
is, you report the modal genotype.

{\bf Answer 2 (quadratic loss; mean):} You decide that, when a mistake is made, some mistakes
are worse than others. For example, if the truth is 0, and you guess 2, then this is worse than
guessing 1, even though both guesses are wrong. A common way to quantify this is via
quadratic loss, which penalizes big mistakes quite harshly:
$$L(\hat{g}; g ) = (g - \hat{g})^2.$$

Now you want to choose $\hat{g}$ to minimize your expected loss
$$E[L(\hat{g};g)] = \sum_g p_g (g-\hat{g})^2.$$
Differentiating with respect to $\hat{g}$, and setting the derivative to zero, yields
$\hat{g} = \sum_g p_g g$. That is, you estimate the genotype by the mean, averaging over the
uncertainty.

Note: The results regarding 0-1 loss and quadratic loss are quite general, and are not limited
only to this example, or to discrete outcomes. That is, reporting a mode of a distribution as
the point estimate is to implicitly adopt 0-1 loss; reporting the mean as a point estimate
is to implicitly adopt quadratic loss.

\item {\it (Proper scoring rules). Assume now that you are asked to report the probability that
the genotypes are 0, 1 or 2. Naively it would appear to make sense to report the probabilities
$p_0, p_1$ and $p_2$. Consider a formal decision-theoretic approach to this problem, with
the following loss functions: i) $L(\hat{p}; g) = 1-\hat{p}_g$; ii) $L(\hat{p};g) = \sum_{i=0}^2 (\hat{p}_i - I(g=i))^2$;
iii) $L(\hat{p};g) = -\log(\hat{p}_g)$.}

Note: loss functions that lead to your reporting your actual probability distribution are referred
to as "proper scoring rules". Which of the above are proper scoring rules? 


\end{enumerate}

\end{document}



